% This is samplepaper.tex, a sample chapter demonstrating the
% LLNCS macro package for Springer Computer Science proceedings;
% Version 2.21 of 2022/01/12
%
\documentclass[a4paper,12pt]{llncs}
%
\usepackage{makeidx}  % allows for indexgeneration
\makeindex
%
%
\usepackage[utf8]{inputenc}
\usepackage[ngerman]{babel}      % chnage the used langauge here!!!
\usepackage[T1]{fontenc}
%
% T1 fonts will be used to generate the final print and online PDFs,
% so please use T1 fonts in your manuscript whenever possible.
% Other font encondings may result in incorrect characters.
%
\usepackage{graphicx}
% correct colors for the hyperrefs
\usepackage[plainpages=false,hypertexnames=true,pdfnewwindow=true,backref=page,colorlinks=true,citecolor=blue,linkcolor=black,urlcolor=blue,filecolor=blue]{hyperref}
%
% additional (maybe useful) packages
\usepackage{amssymb}                % more math symbols
\usepackage{amsmath}                % better equations
\usepackage{booktabs}               % better tables
\usepackage[dvipsnames]{xcolor}     % more colors
\usepackage{spverbatim}             % verbatim with automatic line breaks
\usepackage{dsfont}                 % math fonts
\usepackage{csquotes}               % correct "" based on the used language
\usepackage{listings}               % source code
\usepackage{tikz}                   % custom (vector) graphics
\usetikzlibrary{shapes, arrows.meta, positioning, decorations.pathreplacing, trees, patterns, calligraphy}
%
% pseudo-code
\usepackage{algorithm} 
\usepackage{algpseudocode} 
\newcommand{\algorithmautorefname}{Algorithmus}
%
%
% page format ===============================================================
\hoffset=-1.25truecm
\setlength{\topmargin}{0.0cm}
\setlength{\textheight}{23.0cm}
\setlength{\footskip}{1.5cm}
\setlength{\textwidth}{15.4cm}
\setlength{\evensidemargin}{1.5cm}
\setlength{\oddsidemargin}{1.5cm}
\setlength{\parskip}{1ex}
\setlength{\parindent}{0pt}
\setlength{\marginparwidth}{1.4cm}
\setlength{\marginparsep}{1mm}

\pagestyle{plain}

% LstListing-Format ==========================================================
\lstdefinestyle{cpp}{
  language=C++,
  basicstyle=\small\ttfamily,
  frame=tb,
  xleftmargin=\parindent,
  keywordstyle=\color{blue},
  stringstyle=\color{red},
  commentstyle=\color{ForestGreen},
  framexleftmargin=5pt,
  framexrightmargin=5pt,
  framextopmargin=5pt,
  framexbottommargin=5pt,
  literate={~}{$\sim$}1
}

% macro definitions ==========================================================
% numbers -------------------------------------------------------------
\newcommand{\N}{{\mathbb{N}}}
\newcommand{\R}{{\mathbb{R}}}
\newcommand{\C}{{\mathbb{C}}}
\newcommand{\Z}{{\mathbb{Z}}}
\newcommand{\Q}{{\mathbb{Q}}}
%
%
\def\myverzeichnis{.}
%
\numberwithin{equation}{section}
%
%
% images -----------------------------------------------------------------------
% #1 filename;  #2 Label;  #3 caption;  #4 short-caption
\newcommand{\image}[4]{%
  \begin{figure}[htbp]%
    \begin{center}%
      \includegraphics{#1}%
      \caption[#4]{#3}%
      \label{#2}%
    \end{center}%
  \end{figure}%
}

% image with a specific width -----------------------------------------------------------------
% #1 filename;  #2 width;  #3 Label;  #4 caption;  #5 short-caption
\newcommand{\imagewithwidth}[5]{
  \begin{figure}[htbp]%
    \begin{center}%
      \includegraphics[width=#2]{#1}%
      \caption[#5]{#4}%
      \label{#3}%
    \end{center}%
  \end{figure}
}

% ============================================================================
\begin{document}

% =========== Das war der Vorspann, jetzt geht's los! ========================

% ============================================================================
% =============  AB HIER DARF UND SOLL GETIPPT WERDEN ========================
% ============================================================================
%
\author{Viel Schreiber}
\index{Viel Schreiber}
%
% Das Institut wird fuer den Betreuer missbraucht ...
\institute{{\bf Betreuerin:} Carla Coder}
\authorrunning{Viel Schreiber}
\title{Meine Seminarausarbeitung}
%
%
\maketitle              % typeset the header of the contribution
\thispagestyle{empty}
%
%
\begin{abstract}
Ein schöner Abstract. Das ist einfach die Kurzzusammenfassung.
\end{abstract}


% Introduction -----------------------------------------------------------------
\section{Einleitung}
Lorem ipsum dolor sit amet, consetetur sadipscing elitr, sed diam nonumy eirmod tempor invidunt ut labore et dolore magna aliquyam erat, sed diam voluptua. 
At vero eos et accusam et justo duo dolores et ea rebum. 
Stet clita kasd gubergren, no sea takimata sanctus est Lorem ipsum dolor sit amet. 
Lorem ipsum dolor sit amet, consetetur sadipscing elitr, sed diam nonumy eirmod tempor invidunt ut labore et dolore magna aliquyam erat, sed diam voluptua. 
At vero eos et accusam et justo duo dolores et ea rebum. 
Stet clita kasd gubergren, no sea takimata sanctus est Lorem ipsum dolor sit amet.


\subsection{Anmerkungen zur Einleitung}
Hier kommt noch mehr Text. 
Wir verweisen dazu auf \cite{thisdocument}.

% Formulas -----------------------------------------------------------------
Eine schöne Formel ist
\[ u(\vec{x}) = \sum_{i=1}^N \alpha_i \varphi_i(\vec{x}) \,, \]
aber das geht auch inline als $u(\vec{x}) = \sum_{i=1}^N \alpha_i \varphi_i(\vec{x})$, also mitten im Text.

Es können auch Formeln dargstellt werden, die über mehrere Zeilen gehen, aber trotzdem schön zueinander ausgerichtet sind:
\begin{align*}
    f(x) &= a \cdot (1 + b) \\
         &= a + ab
\end{align*}

% Images/Graphics -----------------------------------------------------------------
Was noch fehlt ist ein Bild, z.B.\ das aus \autoref{fig:grid1} oder \autoref{fig:grid2}. 
Wir können dazu prima die tollen Makros, die oben im Vorspann definiert wurden, verwenden.
Beispielsweise mit folgenden Befehlen:
\begin{spverbatim}
\image{figures/grid_l2.png}{fig:grid1}{Dies ist ein sogenanntes dünnes Gitter zum Level 2.}{Die Kurzform lasse ich meistens leer}
\imagewithwidth{figures/grid_l2.png}{2cm}{fig:grid2}{Dies ist ein sogenanntes dünnes Gitter zum Level 2 in 2cm Breite.}{}
\end{spverbatim}

Die Bilder werden automatisch nach vernünftigen Kriterien platziert, daher immer im Text mit \verb!\autoref{}! drauf verweisen (bei den Beispielen mit \verb!\autoref{fig:grid1}! und \verb!\autoref{fig:grid2}!).
\image{figures/grid_l2.png}{fig:grid1}{Dies ist ein sogenanntes dünnes Gitter zum Level 2.}{Die Kurzform lasse ich meist leer}
\imagewithwidth{figures/grid_l2.png}{2cm}{fig:grid2}{Dies ist ein sogenanntes dünnes Gitter zum Level 2 in 2cm Breite.}{}

% Anmerkung: damit LaTeX nicht denkt, dass ein Punkt den Satz beendet
% (da spendiert LaTeX gerne mehr Zwischenraum), können wir das
% Leerzeichen mit Backslash als Leerzeichen markieren. Damit LaTeX
% ein Leerzeichen setzt, bei dem es keinen Zeilenumbruch geben darf,
% kann man die Tilde verwenden.
% Tables -----------------------------------------------------------------
Was wir hin und wieder noch brauchen ist eine Tabelle, wie z.B.\ \autoref{tab:irgendwas}.
\begin{table}[htbp]
  \centering
  \caption{Diese Tabelle zeigt nicht die Daten von etwas Sinnvollem, sondern einfach irgend etwas. Tabellenbeschriftungen sind oft drüber.}
  \label{tab:irgendwas}
  \begin{tabular}{lrcp{5cm}}
    \toprule
    \multicolumn{3}{c}{Spalten} & Absatz 5cm \\
    \cmidrule(lr){1-3}
    linksbündig & rechtsbündig & zentriert & \\
    \midrule
    1.0 & -1.1 & 1.2 & toller Text, der nach 5cm umbricht und dafür brauchen wir einfach mehr Text. \\
    4321.1 & 6543.2 & 7654.3 & mehr Text \\
    2.44 & 4.66 & 6.88 & 8.00 \\
    \bottomrule
  \end{tabular}
\end{table}


% Source-Code -----------------------------------------------------------------
\subsection{Quellcode}
Code-Beispiele können mittels \texttt{lstlisting}-Environment eingebunden werden.
Siehe \autoref{lst:mylisting} als Beispiel.
Alternativen wie \texttt{minted} sind selbstverständlich auch erlaubt, solange sie Features wie Syntax-Highlighting und Zeilennummern mitbringen.
Code-Beispiele sollten minimal sein, d.h.\ auf den Punkt gebracht und keinen überflüssigen Code beinhalten.
Es muss standardkonformer Code sein und mit hinzugefügtem Boilerplate-Code (main, Auslassungen von Überflüssigem, \dots) ohne Fehler compilierbar sein.

Quellcode aus Dateien kann per \texttt{lstinputlisting} einbezogen werden.
Für Inline-Code \texttt{lstinline} verwenden.
Für abstrakte Algorithmen (kein C++-Code) besser eines der algorithm-Packages verwenden.

\begin{lstlisting}[style=cpp, caption={Example using Lstlisting}, label={lst:mylisting}, numbers=left]
// I'm a comment!
template <typename T>
struct LessThan {
  bool operator(T a, T b) { return a < b; };
};

/*
 * I'm a multiline comment!
 * Ich bin in Kommentar, der mehrere Zeilen verwendet!
 */
std::vector<int> v = { 5, 4, 3, 2, 1 };
std::sort(v.begin(), v.end(), LessThan<int>());

std::cout << "Hello, World" << std::endl;
\end{lstlisting}


% Pseudo-Code -----------------------------------------------------------------
\subsection{Pseudo-Code}
Für Pseudo-Code kann die \texttt{algorithm}-Umgebung verwendet werden.
Ein Beispiel ist in \autoref{alg:ppo} zu sehen.
\begin{algorithm}
	\caption{PPO}
	\begin{algorithmic}[htbp]
		\For {$iteration=1,2,\ldots$}
			\For {$actor=1,2,\ldots,N$}
				\State Run policy $\pi_{\theta_{old}}$ in environment for $T$ time steps
				\State Compute advantage estimates $\hat{A}_{1},\ldots,\hat{A}_{T}$
			\EndFor
			\State Optimize surrogate $L$ wrt. $\theta$, with $K$ epochs and minibatch size $M\leq NT$
			\State $\theta_{old}\leftarrow\theta$
		\EndFor
	\end{algorithmic} 
	\label{alg:ppo}
\end{algorithm}


% Custom graphics -----------------------------------------------------------------
\subsection{Graphiken}
Wenn man einfache Graphiken in seiner Ausarbeitung verwenden will, bietet es sich heirfür immer an, diese selbst zu machen. 
Eine Möglichkeit hierfür ist zum Beispiel TikZ.
Aber Achtung: zu viele TikZ Bilder können die Compile-Zeit von LaTex negativ beeinflussen.

\begin{figure}[htbp]
    \centering
    \begin{tikzpicture}
        \fill[orange] (-0.5, -0.5) rectangle (2.5, 1.5);
        \node[circle, fill=white, draw=black] (A) at (1, 1) {$A$};
        \node[circle, fill=white, draw=black] (B) at (2, 0) {$B$};
        \node[circle, fill=white, draw=black] (C) at (0, 0) {$C$};
        \draw[-to] (A) -- (B);
        \draw[-to] (B) -- (C);
        \draw[-to] (C) -- (A);
    \end{tikzpicture}
    \caption{Ein einfaches Beispiel eines Graphen gezeichnet via TikZ.}
    \label{fig:tikz}
\end{figure}

\subsection{Zum Schluss}
\dots viel Spaß!


% ---- Bibliography ----
\bibliographystyle{alpha}
\bibliography{literatur.bib}

\end{document}

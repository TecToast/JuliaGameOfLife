% This is samplepaper.tex, a sample chapter demonstrating the
% LLNCS macro package for Springer Computer Science proceedings;
% Version 2.21 of 2022/01/12
%
\documentclass[a4paper,12pt]{llncs}
%
\usepackage{makeidx}  % allows for indexgeneration
\makeindex
%
%
\usepackage[utf8]{inputenc}
\usepackage[english]{babel}      % chnage the used langauge here!!!
\usepackage[T1]{fontenc}
%
% T1 fonts will be used to generate the final print and online PDFs,
% so please use T1 fonts in your manuscript whenever possible.
% Other font encondings may result in incorrect characters.
%
\usepackage{graphicx}
% correct colors for the hyperrefs
\usepackage[plainpages=false,hypertexnames=true,pdfnewwindow=true,backref=page,colorlinks=true,citecolor=blue,linkcolor=black,urlcolor=blue,filecolor=blue]{hyperref}
%
% additional (maybe useful) packages
\usepackage{amssymb}                % more math symbols
\usepackage{amsmath}                % better equations
\usepackage{booktabs}               % better tables
\usepackage[dvipsnames]{xcolor}     % more colors
\usepackage{spverbatim}             % verbatim with automatic line breaks
\usepackage{dsfont}                 % math fonts
\usepackage{csquotes}               % correct "" based on the used language
\usepackage{listings}               % source code
\usepackage{tikz}                   % custom (vector) graphics
\usetikzlibrary{shapes, arrows.meta, positioning, decorations.pathreplacing, trees, patterns, calligraphy}
%
% pseudo-code
\usepackage{algorithm} 
\usepackage{algpseudocode} 
\newcommand{\algorithmautorefname}{Algorithmus}
%
%
% page format ===============================================================
\hoffset=-1.25truecm
\setlength{\topmargin}{0.0cm}
\setlength{\textheight}{23.0cm}
\setlength{\footskip}{1.5cm}
\setlength{\textwidth}{15.4cm}
\setlength{\evensidemargin}{1.5cm}
\setlength{\oddsidemargin}{1.5cm}
\setlength{\parskip}{1ex}
\setlength{\parindent}{0pt}
\setlength{\marginparwidth}{1.4cm}
\setlength{\marginparsep}{1mm}

\pagestyle{plain}

% LstListing-Format ==========================================================
\lstdefinestyle{cpp}{
  language=C++,
  basicstyle=\small\ttfamily,
  frame=tb,
  xleftmargin=\parindent,
  keywordstyle=\color{blue},
  stringstyle=\color{red},
  commentstyle=\color{ForestGreen},
  framexleftmargin=5pt,
  framexrightmargin=5pt,
  framextopmargin=5pt,
  framexbottommargin=5pt,
  literate={~}{$\sim$}1
}

% macro definitions ==========================================================
% numbers -------------------------------------------------------------
\newcommand{\N}{{\mathbb{N}}}
\newcommand{\R}{{\mathbb{R}}}
\newcommand{\C}{{\mathbb{C}}}
\newcommand{\Z}{{\mathbb{Z}}}
\newcommand{\Q}{{\mathbb{Q}}}
%
%
\def\myverzeichnis{.}
%
\numberwithin{equation}{section}
%
%
% images -----------------------------------------------------------------------
% #1 filename;  #2 Label;  #3 caption;  #4 short-caption
\newcommand{\image}[4]{%
  \begin{figure}[htbp]%
    \begin{center}%
      \includegraphics{#1}%
      \caption[#4]{#3}%
      \label{#2}%
    \end{center}%
  \end{figure}%
}

% image with a specific width -----------------------------------------------------------------
% #1 filename;  #2 width;  #3 Label;  #4 caption;  #5 short-caption
\newcommand{\imagewithwidth}[5]{
  \begin{figure}[htbp]%
    \begin{center}%
      \includegraphics[width=#2]{#1}%
      \caption[#5]{#4}%
      \label{#3}%
    \end{center}%
  \end{figure}
}

% ============================================================================
\begin{document}

% =========== Das war der Vorspann, jetzt geht's los! ========================

% ============================================================================
% =============  AB HIER DARF UND SOLL GETIPPT WERDEN ========================
% ============================================================================
%
\author{Florian Schröder}
\index{Florian Schröder}
%
% Das Institut wird fuer den Betreuer missbraucht ...
\institute{{\bf Supervisor:} Gerasimos Chourdakis}
\authorrunning{Viel Schreiber}
\title{Can Julia win the Game of Life?}
%
%
\maketitle              % typeset the header of the contribution
\thispagestyle{empty}
%
%
\begin{abstract}
  This paper gives an overview of Conway's Game of Life (and cellular automata in general) and various implementations of it in the Julia programming language.
  It is mostly based on Daniel Shiffman's Nature of Code \cite{NOC}.
\end{abstract}


% Introduction -----------------------------------------------------------------
\section{Introduction}

\section{Methodology}


\section{Results}
\subsection{Understanding cellular automata (1D)}
Grid, states, neighborhood, function which creates the new state
\subsection{Implementation of 1D in Julia}
Here I'll also include specific design decisions and discuss the implications of these choices.
\subsection{Jump to 2D cellular automata (Game of Life)}
The theory of moving to 2D (now where displaying the "history" by using an animation)
\subsection{Implementation of 2D (Game of Life) in Julia}
Here I'll differentiate between simple solutions (like a bool matrix for storing)
and an object-oriented approach which may be even complexer
\subsection{Various interesting Game of Life patterns}
Stable/oscillating/moving patterns which can be destroyed by a single tile (which shows how sensitive such a system might be)
\subsection{Comparison with other programming languages}
Here I compare the performance with languages, I think I'll choose JavaScript and Python.
I don't think that I'll provide implementations with these languages, rather a brief overview of them.
\section{Discussion}
\section{Conclusions}
Here I will summarize the findings and implications of the research conducted.

% ---- Bibliography ----
\bibliographystyle{alpha}
\bibliography{literatur.bib}

\end{document}
